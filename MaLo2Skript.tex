% !TXS template
\documentclass[german]{article}
\usepackage[T1]{fontenc}
\usepackage[utf8]{inputenc}
\usepackage{lmodern}
\usepackage[a4paper]{geometry}
\usepackage{babel}

\usepackage{amsmath}
\usepackage{amsfonts}
\usepackage{amsthm}

\usepackage{mathtools}

\usepackage{tikz}

\title{Mathematische Logik 2:\\Vorlesungmitschrift}
\author{Theodor Teslia}
\date{}


\newtheoremstyle{break}
{\topsep}{\topsep}%
{}{}%
{\bfseries}{}%
{\newline}{}%
\theoremstyle{break}
\newtheorem{example}{Beispiel}

\newtheoremstyle{def_style}
{\topsep}{\topsep}%
{}{}%
{\bfseries}{}%
{\newline}{}%
\theoremstyle{def_style}
\newtheorem{definition}{Definition}

\newtheoremstyle{def_style}
{\topsep}{\topsep}%
{}{}%
{\bfseries}{}%
{\newline}{}%
\theoremstyle{def_style}
\newtheorem{satz}{Satz}


\begin{document}

\maketitle

\section{Mengenlehre}

\glqq Aus dem Paradies, das Cantor uns geschaffen, soll uns niemand vertreiben können.\grqq{}\\
(David Hilbert, 1926) 

\subsection{Mengen und Klassen}

Reine Mengen sind Kollektion an Objekten, die ebenfalls wieder Mengen sind, als Alternative lassen sich Mengen ausgehend von Urelementen definieren.
\\
Wie führt die Mathematik Objekte ein?
\begin{itemize}
	\item Explizite Konstruktion aus schon vorhandenen Objekten, bspw. Konstruktion der rationalen, reellen und komplexen Zahlen, ausgehend von den ganzen und natürlichen.
	\item Axiomatisches formulieren von gewünschten Eigenschaften der Objekte und betrachte alle Objekte, die die Eigenschaften erfüllen, bspw. Gruppen, Vectorräume, ...
\end{itemize}

\begin{definition}[Mengen]
	Intuitiv sind Mengen Kollektionen von Objekten, die selbst wieder Mengen sind. 
	
	$a\in b$: $a$ ist ein Element in der Menge $b$.
	
	$a \subseteq b$: Jedes Element von $a$ ist auch in $b$.
	\\
	Eine Konstruktive Definition von aller Menge könnte wie folgt aussehen:\\ $\{\emptyset\}, \{\emptyset,\{\emptyset\}\}, \{\{\emptyset\}\}$ \textit{usw.}
	Problem: Wie sieht dieses \textit{usw.} aus?
\end{definition}


Mengen lassen sich auch als Bäume darstellen. Dies lässt sich in Abbildung \ref{Mengenbaum} Beispielhaft für die Menge $\{\; \emptyset, \; \{\emptyset\}, \; \{\emptyset,\{\emptyset\}\}\;\}$ erkennen. 

\begin{figure}[h]
	\begin{center}
		\begin{tikzpicture}
			\node {$\bigcirc$}
			child {node {$\emptyset$}}
			child {node {$\bigcirc$}
				child {node {$\emptyset$}}}
			child {node {$\bigcirc$}
				child {node {$\emptyset$}}
				child {node {$\bigcirc$}
					child {node {$\emptyset$}}}};
		\end{tikzpicture}
	\end{center}
\caption{Darstellung der Menge $\{\; \emptyset, \; \{\emptyset\}, \; \{\emptyset,\{\emptyset\}\}\;\}$ als Baum}
\label{Mengenbaum}
\end{figure}

\begin{definition}[Hereditär endliche Mengen]
	Die hereditär endlichen Mengen ($HF$) bilden eine Miniaturversion der Mengenlehre. Es gilt $HF_0\subset HF_1 \subset HF_2 \subset \dots$. Definiert sind diese Mengen induktiv als $HF_0\coloneqq \emptyset$ und $HF{n+1}\coloneqq \{x : x\subseteq HF_n\}$, so dass $HF_{n+1}$ die Potenzmenge von $HF_n$ ist.
	
	Eine Menge ist hereditär endlich, wenn sie Element einer Menge $HF_n$ für ein $n$ ist. Weiter wird $HF\coloneqq\{x : x \in HF_n \text{ für ein }n\in \mathbb{N}\}$ definiert.
	Dies wirft folgende Frage auf: Ist $HF$ eine Menge? 
\end{definition}

Die ersten $HF$ Mengen lauten $HF_0=\emptyset$, $HF_1=\{\emptyset\}$, $HF_2=\{\emptyset,\{\emptyset\}\}$, $HF_3=\{\emptyset, \{\emptyset\}, \{\{\emptyset\}\}, \{\emptyset,\{\emptyset\}\}\}$. Es lässt sich erkennen, dass $HF_n\subset HF_{n+1}$ und $HF_n\in HF_{n+1}$ für ein beliebiges $n$. Auch ist es möglich zu folgern, dass $HF_n$ endlich viele Elemente besitzt und jede Menge $a\in HF_{n+1}$ die Gestalt $a=\{b_0,\dots,b_{k-1}\}$ mit $b_0,\dots,b_{k-1}\in HF_n$. Außerdem gilt, dass $HF$ nicht hereditär endlich ist.

\begin{definition}[Natürliche Zahlen]
	Eine natürliche Zahl $n$ ist definiert als $[n]\coloneqq\{[0],\dots,[n-1]\}$, wobei $[0]=\emptyset$ gilt. Die Menge der natürlichen Zahlen $\mathbb{N}$ lässt sich nun als $\mathbb{N}\coloneqq\{[n] : n \text{ eine nat. Zahl}\} \notin HF$. Eine Folgerung ist $[n]\in HF_{n+1}\setminus HF_n$.
\end{definition}

\subsubsection{Axiomensysteme für die Mengenlehre}

Das Modell der Mengenlehre besteht aus einer Kollektion $S$ von Objekten, die wir Mengen nennen und einer Beziehung $\in$ zwischen diesen Objekten, so dass alle Axiome des Axiomensystems erfüllt werden.

\begin{definition}[Extensionalitätsaxiom (Ext.)]
	Zwei Mengen sind gleich, genau dann, wenn sie die selben Elemente haben. In einer Prädikatenlogischen-Formel mit Signatur $\{\in\}$ wäre dies $\forall x \forall y (x=y \leftrightarrow \forall z (z \in x \leftrightarrow z \in y))$.
	\label{ExtAxiom}
\end{definition}

Die \textit{Konsistenz} des Axiomensystems der Mengenlehre beschreibt, ob es ein Modell des Axiomensystems gibt oder ob dieses Widersprüchlich ist. Es ist nicht möglich, die Konsistenz unseres Axiomensystems zu beweisen.

Das Axiomensystem der Mengenlehre ist \textit{Vollständig}, wenn alle Modelle \glqq gleich \grqq{} sind, in dem Sinn, dass die gleichen Eigenschaften gelten. Es gibt kein vollständiges Axiomensystem für die Mengenlehre.
\\
\\
Nehmen wir an, dass $(S,\in)$ ein beliebiges, aber festes Modell der Mengenlehre ist. Die Axiome sollen regeln, welche Kollektionen von Elementen aus $S$ selbst wieder ELemente von $S$, also Mengen sind. Dabei werden Kollektionen \textit{Klassen} genannt und Klassen, die keine Mengen sind, werden als \textit{echte Klassen} bezeichnet.

\begin{definition}[Die naive Mengenlehre]
	Das Axiomensystem der naiven Mengenlehre besitzt zwei Axiome. Zum einen das Extensionalitätsaxiom (siehe Definition \ref{ExtAxiom}) und das Axiomenschema der vollen Komprehension. Dieses besagt, dass sich für jede Formel $\psi(x)$ die Menge $\{x : \psi(x)\}$ bilden lässt: $\exists z \forall z(x \in z \leftrightarrow \psi(x))$.
\end{definition}

\begin{satz}[Zermelo-Russel Paradoxon]
	Die naive Mengenlehre ist inkonsistent.
	
	Sei $\psi(x)\coloneqq x\notin x$. Nach dem Komprehensionsschema muss nun folgende Formel gelten: $\exists z \forall x(x\in z \leftrightarrow x\notin x)$, aus welcher die Menge $z=\{x : x\notin x\}$ folgt. Eine solche Menge kann aber nicht existieren, da sonst $z\in z \Leftrightarrow z \notin z$ gelten müsste. Es folgt, dass $\{x:x\notin x\}$ immer einer echte Klasse sein muss.
\end{satz}

\begin{definition}[Das Axiomensystem ZFC]
	Das Axiomensystem ZFC (\textbf{Z}ermelo-\textbf{F}raenkel-\textbf{C}hoice) ist das heutzutage benutzte Axiomensystem. Es besitzt die folgenden Axiome:
	\begin{itemize}
		\item Das Extensionalitätsaxiom
		\item Das Aussonderungsaxiom: $\forall z \exists y \forall x (x\in y \leftrightarrow(x\in z \land \psi(x)))$. Dieses ist eine schwächere Version des Komprehensionsschemas, bei dem eine Menge aus bereits bestehenden Menge ausgewählt wird.
		\item Das Erzeugungsaxiom (Kreationsaxiom): Für jede Menge $x$ ex. eine \textit{Stufe} $s\in S$ mit $x\in s$.
		\item Das Unendlichkeitsaxiom: Es gibt eine \textit{Limesstufe} und damit eine unendliche Menge.
		\item Das Ersetzungsaxiom: Für jede Funktion $F:S\to S$ mit der Eigenschaft, dass $Def(F)$ eine Menge ist, ist auch $Bild(F)$ eine Menge.
		\item Das Auswahlaxiom: Auf jeder Menge ex. eine \textit{Auswahlfunktion}
	\end{itemize}
\end{definition}

\begin{definition}[Klassenoperatoren]
	Seien $A, B$ Klassen.
	
	$A \subseteq B$: Jede Menge aus $A$ ist auch in $B$.
	
	$\bigcap A\coloneqq\{x : x\in y \text{ für alle } y \in A\}$
	
	$A \cap B \coloneqq \{x : x\in A \text{ und } x \in B\}$
	
	$A \setminus B \coloneqq \{x : x \in A \text{ aber } x \notin B\}$
\end{definition}

Für eine Formel $\psi(x)$ kann $\{x : \psi(x)\}$ entweder eine Menge oder eine echte Klasse sein. Somit lässt sich das Aussonderungsaxiom umformulieren: Für jede Menge $a$ und jede Klasse $A$ ist $a \cap A$ eine Menge. Daraus folgt, dass auch $a \setminus A$ eine Menge sein muss, dass $\bigcap A$ eine Menge ist, falls $A$ mindestens eine Menge enthält und, dass $\bigcap \emptyset = S$ keine Menge ist.

\subsection{Stufen und Geschichten}

Eine mögliche Methode zur Definition der gesamten Klasse aller Mengen ist es, die induktive Konstruktion der $HF$ zu erweitern. So ist $S$ dann die Vereinigung einer aufsteigenden Folge von Mengen $S_\alpha$, welche wir die Stufen von $S$ nennen. Es gilt $S_0\coloneqq \emptyset$ und für die bereits definierte Stufe $S_\alpha$ setzen wir $S_{\alpha +1}\coloneqq\{x : x\in S_\alpha\}$.

Sobald eine unendliche Folge von solchen Stufen definiert wurde, lässt sich eine neue Stufe als Vereinigung aller bisherigen Stufen definieren.\\

$S_0=HF_0=\emptyset$, $S_1=HF_1$, $\dots$, $S_n=HF_n$

$S_\omega \coloneqq \bigcup\limits_n S_n = \bigcup\limits_n HF_n = HF$

$S_{\omega+1}\coloneqq\{x : s\in S_\omega\}$, $\dots$

$S_{\omega+\omega}\coloneqq\{x : x \in S_{\omega+n} \text{ für ein } n\}$ \textit{usw.}

\begin{definition}[Die Akkumulation]
	Sei $A$ eine Klasse. Die Akkumulation von $A$ ist $acc(A)\coloneqq\{x : (\exists y \in A) x\in y\lor x \subset y\})$.
\end{definition}

Da für eine Klasse $A$ und ein Element $a\in A$ natürlich gilt, dass $a \subset a$, gilt auch $a\in acc(A)$ und somit $A\subseteq acc(A)$.

\begin{definition}[Geschichten]
	Eine Geschichte ist eine Klasse $H$, so dass für alle $a\in H$ gilt $acc(a\cap H)=a$.
\end{definition}

Die Stufe mit Geschichte $H$ ist $S\coloneqq acc(H)$.

\begin{example}[Beispiele zu Geschichten und Stufen]
	Im Folgenden soll für einige Mengen ihre Akkumulation gezeigt werden und bewiesen, dass es sich bei diesen auch um Geschichten handelt.
	\begin{itemize}
		\item $acc(\emptyset)=\emptyset$. $\emptyset$ ist eine Geschichte und die Stufe mit Geschichte $\emptyset$ ist $\emptyset$.
		\item $\{\emptyset\} = [1] = HF_1 = \{HF_0\}$: $acc(\{\emptyset\}) = \{\emptyset\}$. 
		$\{\emptyset\}$ ist auch eine Geschichte, denn für das einzige Element $\emptyset$ gilt $acc(\emptyset \cap \{\emptyset\})=acc(\emptyset)=\emptyset$. 
		Die Stufe mit Geschichte $\{\emptyset\}$ ist $\{\emptyset\}$.
		\item $\{\emptyset, \{\emptyset\}\}=[2]=HF_2=\{HF_0, HF_1\}$: $acc(HF_2)=HF_2$.
		$HF_2$ ist eine Geschichte, denn $acc(\emptyset \cap \{\emptyset, \{\emptyset\}\})=acc(\emptyset)=\emptyset$ und $acc(\{\emptyset\}\cap \{\emptyset, \{\emptyset\}\})=acc(\{\emptyset\})=\{\emptyset\}$. Die Stufe mit Geschichte $\{\emptyset, \{\emptyset\}\}$ ist $\{\emptyset, \{\emptyset\}\}$.
	\end{itemize}
\label{GeschichtenBsp}
\end{example}

Dies wirft die Frage auf, ob es eine Verallgemeinerung gibt. Demnach soll nun überprüft werden, ob $[n]$ eine Geschichte ist, für jedes $n$ in den natürlichen Zahlen. Für $k<n$ müsste gelten, dass $acc([n]\cap[k])=acc([k])\stackrel{!}{=}[k]$. 
Aber $acc([k])$ enthält alle Teilmengen von $[k-1]$, für $k=4$ also alle Teilmengen von $\{[0],[1],[2]\}$, demnach auch $\{[0], [2]\}$. Es gilt aber, dass $\{[0],[2]\}\notin [k]$ und es folgt $acc([k])\neq[k]$. Für $n \geq 3$ ist $n$ also keine Geschichte.

Ist $HF_n$ eine Geschichte? Nein, denn $[n-1]\in HF_n$ und $acc([n-1]\cap HF_n)=acc([n-1])\neq[n-1]$.

Aber $G_n\coloneqq\{HF_0,\dots,HF_{n-1}\}$ ist eine Geschichte mit Stufe $HF_n$. Für $n=0,1,2$ wurde dies schon in Beispiel \ref{GeschichtenBsp} gezeigt. Sei dies für $G_n$ bereits bewiesen, wir zeigen dies nun für $G_{n+1}=G_n\cup\{HF_n\}$.
$G_{n+1}$ ist eine Geschichte, wenn für alle $k \leq n$ gilt: $acc(HF_k\cap G_{n+1})=HF_k$.
$HF_k \cap G_{n+1} = \{HF_0,\dots,HF_{k-1}\}=G_k$ und per Induktionsvoraussetzung gilt $acc(G_k)=HF_k$.

Also ist $G_{n+1}$ eine Geschichte. Die Stufe mit Geschichte $G_{n+1}$ ist $acc(G_{n+1})=acc(G_n\cup \{HF_n\})=acc(GF_n)\cup HF_n \cup \{x : x\subseteq HF_n\}=HF_n\cup HF_n \cup HF_{n+1} = HF_{n+1}$.

Dies gibt die Idee für die Rückrichtung: Für jede Stufe $S_\alpha$ soll gelten, dass sie die Geschichte $H(S_\alpha)=\{S_\beta : \beta < \alpha\}$ hat.

\begin{definition}[Minimales Element]
	Eine Menge $m\in A$ ist ein \textit{minimales Element} von $A$, wenn $m\cap A=\emptyset$, d.h. es gibt kein $a\in A$ mit $a\in m \in A$.
	
	Eine Menge $a$ ist fundiert, wenn jede Menge $b$ mit $a\in b$ ein minimales Element enthält. Der fundierte Teil von $A$ ist $fd(A)\coloneqq\{x\in A : x \text{ ist fundiert}\}$.
\end{definition}

\begin{example}[Beispiele für minimale Elemente und Fundiertheit]
	$\emptyset$ ist fundiert.
	
	$\{\emptyset\}$ ist ebenfalls fundiert. Sei $\{\emptyset\}\in b$. Wenn $\{\emptyset\}\cap b =\emptyset$ ist $\{\emptyset\}$ das minimale Element. Andernfalls ist $\{\emptyset\}\cap b=\{\emptyset\}$ und $\emptyset$ ist das minimale Element von $b$.
\end{example}

\begin{satz}[Wenn $H$ eine Geschichte ist, dann enthält jede nicht-leere Teilmenge von $H$ ein minimales Element]
	Es sei $a \in b \subset H$ 
\end{satz}










\end{document}
